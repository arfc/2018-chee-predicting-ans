\documentclass{anstrans}
%%%%%%%%%%%%%%%%%%%%%%%%%%%%%%%%%%%
\title{Validation of Isotopic Composition of Spent Nuclear Fuel output by Cyclus, a Fuel Cycle Simulator Code}
\author{Gwendolyn J. Chee, Gyutae Park, and Kathryn D. Huff}

\institute{
Dept. of Nuclear, Plasma and Radiological Engineering, University of Illinois at Urbana-Champaign \\
gchee2@illinois.edu
}

%%%% packages and definitions (optional)
\usepackage{graphicx} % allows inclusion of graphics
\usepackage{booktabs} % nice rules (thick lines) for tables
\usepackage{microtype} % improves typography for PDF
\usepackage{xspace}
\usepackage{tabularx}
\newcommand{\SN}{S$_N$}
\renewcommand{\vec}[1]{\bm{#1}} %vector is bold italic
\newcommand{\vd}{\bm{\cdot}} % slightly bold vector dot
\newcommand{\grad}{\vec{\nabla}} % gradient
\newcommand{\ud}{\mathop{}\!\mathrm{d}} % upright derivative symbol
\newcommand{\Cyclus}{\textsc{Cyclus}\xspace}%
\newcommand{\Cycamore}{\textsc{Cycamore}\xspace}%
\newcolumntype{c}{>{\hsize=.56\hsize}X}
\newcolumntype{b}{>{\hsize=.7\hsize}X}
\newcolumntype{s}{>{\hsize=.74\hsize}X}
\newcolumntype{f}{>{\hsize=.1\hsize}X}
\newcolumntype{a}{>{\hsize=.45\hsize}X}

\begin{document}
%%%%%%%%%%%%%%%%%%%%%%%%%%%%%%%%%%%%%%%%%%%%%%%%%%%%%%%%%%%%%%%%%%%%%%%%%%%%%%%%
\section{Introduction}
\Cyclus \cite{huff_fundamentals_2016}, a fuel cycle simulator, was used to run a simulation of the
United States' nuclear fuel cycle. Validation was performed using the resulting isotopic
concentration of spent fuel with respect to spent nuclear fuel (SNF) data from the U.S Department of
Energy (DOE) sponsored Unified Database (UDB) \cite{Peterson_UNF_2017}.  UDB provides
comprehensive and consistent technical data on reactor sites and spent nuclear fuel (SNF) up to the
year 2013.

The cumulative mass of spent nuclear fuel and major isotopes were compared. Comparison of the
\Cyclus output and UDB data provides baseline comparison models between \Cyclus simulated
results and real world metrics.

%%%%%%%%%%%%%%%%%%%%%%%%%%%%%%%%%%%%%%%%%%%%%%%%%%%%%%%%%%%%%%%%%%%%%%%%%%%%%%%%
\section{Motivation}
The United States is considering various fuel cycles and geologic disposal options
\cite{DOE_strategy_2013}. Key criteria such as thermal capacity and load influence the decisions
made as various types of geologic host media have different site geology thermal parameters
\cite{huff_dynamic_2013}. Geologic repository modeling requires accurate waste package thermal
load data to accurately model thermal distribution. Accurate thermal load data is dependent on
accurate isotopic composition of spent fuel. 



%%%%%%%%%%%%%%%%%%%%%%%%%%%%%%%%%%%%%%%%%%%%%%%%%%%%%%%%%%%%%%%%%%%%%%%%%%%%%%%%
\section{Methodology}
\noindent 
\textit{Generating \Cyclus Simulation of United States Nuclear Fuel Cycle}

\noindent 
\textit{\Cyclus Data Analysis}

\noindent 
\textit{UDB Data Analysis}




%%%%%%%%%%%%%%%%%%%%%%%%%%%%%%%%%%%%%%%%%%%%%%%%%%%%%%%%%%%%%%%%%%%%%%%%%%%%%%%%
\section{Results and Analysis}


%%%%%%%%%%%%%%%%%%%%%%%%%%%%%%%%%%%%%%%%%%%%%%%%%%%%%%%%%%%%%%%%%%%%%%%%%%%%%%%%
\section{Conclusions}

%%%%%%%%%%%%%%%%%%%%%%%%%%%%%%%%%%%%%%%%%%%%%%%%%%%%%%%%%%%%%%%%%%%%%%%%%%%%%%%%
\section{Acknowledgments}
This research is being performed using funding received from the DOE Office of Nuclear Energy's
Nuclear Energy University Program (Project 16-10512) "Demand-Driven Cycamore Archetypes". 

%%%%%%%%%%%%%%%%%%%%%%%%%%%%%%%%%%%%%%%%%%%%%%%%%%%%%%%%%%%%%%%%%%%%%%%%%%%%%%%%
\bibliographystyle{ans}
\bibliography{bibliography}
\end{document}

